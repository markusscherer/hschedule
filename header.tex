\documentclass[landscape]{article}
\usepackage{tikz}
\usepackage[cm]{fullpage}
\usepackage[utf8]{inputenc}
\usepackage{varwidth}

\renewcommand{\familydefault}{\sfdefault}
\pagenumbering{gobble}

\begin{document}

% These set the width of a day and the height of an hour.
\newcommand*\daywidth{4.5cm}
\newcommand*\hourheight{3.5em}

% This style is used for entries without a explicit style
\tikzset{entrystyle/.style 2 args={
    fill=gray!30,
    rectangle,
    anchor=north west,
    align=left,
    inner sep=0.2em,
    text width={\daywidth/#2-0.4em},
    minimum height=#1*\hourheight,
}}

% explicit style .AStyle
\tikzset{entrystyleAStyle/.style 2 args={
    entrystyle={#1}{#2},
    fill=blue!25,
}}

% explicit style .BStyle
\tikzset{entrystyleBStyle/.style 2 args={
    entrystyle={#1}{#2},
    fill=cyan!25,
}}

% explicit style .CStyle
\tikzset{entrystyleCStyle/.style 2 args={
    entrystyle={#1}{#2},
    fill=orange!25,
}}

% This style is used for the times on the left side.
\tikzset{timestyle/.style={
  draw,
  fill=green!20,
  anchor=north east,
  outer sep=0.0pt,
  minimum height=\hourheight
}}

% This style is used for the headers on the top
\tikzset{headerstyle/.style={
  draw,
  fill=yellow!10,
  anchor=north,
  minimum width=\daywidth,
  minimum height=\hourheight
}}

% This command is used for entries with 3 parameters (including start and end time)
\newcommand{\entryIII}[3] {
      {\small{#1} - \small{#2}} \linebreak
      {\textbf{#3}} \linebreak
}

% This command is used for entries with 4 parameters (including start and end time)
\newcommand{\entryIV}[4] {
      {\small{#1} - \small{#2}} \linebreak
      {\textbf{#3}} \linebreak
      {#4}
}

% This command is used for entries with 5 parameters (including start and end time)
\newcommand{\entryV}[5] {
      {\small{#1} - \small{#2}} \linebreak
      {\textbf{#3}} \linebreak
      {#4 \hfill #5}
}

\newcommand{\timeentry}[2] {
  \large{#1 - #2}
}

\newcommand{\headerentry}[1] {
  \textbf{\large{#1}}
}

% Start the picture and set the x coordinate to correspond to days and the y
% coordinate to correspond to hours (y should point downwards).
\begin{tikzpicture}[y=-\hourheight,x=\daywidth]
